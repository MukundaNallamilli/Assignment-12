\documentclass{beamer}

\usepackage{amssymb}
\usepackage{amsfonts}
\usepackage{amsmath}
\usepackage{amsthm}
\usepackage{setspace}
\usepackage{longtable}
\usepackage{graphicx}
\usepackage{mathtools}
\usepackage{color}
\usepackage{array}
\usepackage{calc} 
\usepackage{bm}
\usepackage{caption}
\usepackage{float}

\usetheme{CambridgeUS}
\useoutertheme{infolines}
%numbering
\setbeamercolor{background canvas}{bg=white}
\setbeamersize{text margin left=1cm,text margin right=1cm}

\title[AI1110  Assignment-12]{ASSIGNMENT-12}
\subtitle{AI1110}
\author[]{MUKUNDA REDDY \\ AI21BTECH11021}
\date

\begin{document}
  \begin{frame}
      \titlepage
  \end{frame}
  
  \begin{frame}{Outline}
      \tableofcontents
  \end{frame}
  
  \section{Question}
  \begin{frame}{Exercise 15-15}
   Deteimine the mean time to absorption for the random walk
   model  with two absorbing barriers .Let the number of
   states in a random walk be finite $(e_0, e_1,... e_N)$
   Consider the case 
   \begin{center}
   P =
   \begin{pmatrix}
    1 &0& 0& 0& .& .& .& 0\\
    q &0& p& 0& 0& .& .& 0 \\
      0& q& 0& P& 0& .& .& 0 \\
       .&.&.&.&.&.&.&.   \\
       .&.&.&.&.&.&.&. \\
       0& 0& .& .& .& q& 0& P  \\
       0& 0& .& .& .& 0& 0& 1&
   \end{pmatrix}
   \end{center}
    Find mean time to absorption for player A?
  \end{frame}
  
  \section{Solution}
  \begin{frame}{Solution}
      The mean time to absorption satisfies
      \begin{align*}
          m_i &= 1 + \sum_{k \in T} p_{ik}m_k \\
             &= 1 + p_{i,i+1}m_{i+1} + p_{i,i-1}m_{i-1} \\
             &= 1 + pm_{i+1} + qm_{i-1}
      \end{align*}
      This gives 
      $$ p(m_{k+1} - m_{k}) = q(m_k - m_{k-1}) - 1$$
      Let
      $$ M_{k+1} = m_{k+1} - m_{k} $$
  \end{frame}
  
  \begin{frame}{Solution}
      so that the above iteration gives
      \begin{align*}
          M_{k+1} &= \frac{q}{p}M_{k} - \frac{1}{p} \\
                 &= \left(\frac{q}{p}\right)^{k}M_1 - \frac{1}{p}\left[ 
                 1 + \left(\frac{q}{p}\right) + \left(\frac{q}{p}\right)^2 + ... +
                 \left(\frac{q}{p}\right)^{k-1} \right] \\
            &=
            \begin{cases}
                 {\left(\frac{q}{p}\right)}^{k}M_1 - \frac{1}{p-q} \left\{ 1-{\left(\frac{q}{p}\right)}^{k} \right\} ,& p \neq q \\
                 M1 - \frac{k}{p}, & p=q 
            \end{cases} \\
      \end{align*}
  \end{frame}
  
  \begin{frame}{Solution}
      This gives
      \begin{align*}
          m_i &= \sum_{k=0} ^{i-1} M_{k+1} \\
            &=
            \begin{cases}
               \left(M_1 + \frac{1}{p-q}\right) \sum_{k=0} ^{i-1} {\left(\frac{q}{p}\right)}^k - \frac{i}{p-q}, & p \neq q \\
               iM_1 - \frac{i(i-1)}{2p}, & p=q
            \end{cases} \\
            &=
            \begin{cases}
                \left(M_1 + \frac{1}{p-q}\right)\frac{1-(q/p)^{i}}{1-(q/p)} - \frac{i}{p-q}, & p \neq q \\
                iM_1 - \frac{i(i-1)}{2p}, & p=q 
            \end{cases}
      \end{align*}
  \end{frame}
  
  \begin{frame}{Solution}
      which we have used $m_o =0$.Similarly $m_{a+b} = 0$ gives
      $$M_1 + \frac{1}{p-q} = \frac{a+b}{p-q}\frac{1- q/p}{1 - (q/p)^{a+b}}.$$
      Thus 
      $m_i =$
      \begin{cases}
          \frac{a+b}{p-q}\frac{1-(q/p)^i}{1-(q/p)^{a+b}}- \frac{i}{p-q}, & p\neq q \\
          i(a+b-i), & p=q 
      \end{cases}
  \end{frame}
  
  \begin{frame}{Solution}
      Which gives for $i=a$
      \begin{align*}
          m_a &=
          \begin{cases}
              \frac{a+b}{p-q} \frac{1-(q/p)^{a}}{1-(q/p)^{a+b}} - \frac{a}{p-q}, & p \neq q \\
              ab, & p=q
          \end{cases} \\
          &= 
          \begin{cases}
                \frac{b}{2p-1} - \frac{a+b}{2p-1}\frac{1-(p/q)^{b}}{1-(p/q)^{a+b}}, & p \neq q \\
                ab, & p=q
          \end{cases} \\
      \end{align*}
  \end{frame}
  
  \begin{frame}{Solution}
      by writing we get the mean time as
      \begin{align*}
          \frac{1-(q/p)^a}{1 - (q/p)^{a+b}} &= 1 - \frac{(q/p)^{a} - (q/p)^{a+b}}{1 - (q/p)^{a+b}} \\
          &= 1 - \frac{1-(p/q)^b}{1 - (p/q)^{a+b}}
      \end{align*}
  \end{frame}
  
\end{document}
